\section{Hasse Diagrams}
Finite partial orders can be represented using a mathematical diagram called a \emph{Hasse diagram}.

Vertices (often labelled) represent elements of $U$, and for every $x\sqle{}y$
such that there is no $z$ such that $x\sqle{}z\sqle{}y$, there is a line which
goes upward from $x$ to $y$.

We assume transitivity, i.e. if $x\sqle{}y$ and $y\sqle{}z$ according to the diagram, then we can deduce that $x\sqle{}z$.

We also assume reflexivity, i.e. $x \sqle{} x$ for every $x$ in the diagram.

\begin{center}
  \begin{tikzcd}[tips=false]
    A \ar{d} &&&& B \\
    C \ar{dr} && D \ar{dl} \ar{dr} \\
    & E && F
  \end{tikzcd}
\end{center}

In the above diagram, $E \sqle{} A$ because $E \sqle{} C$ and $C \sqle{} A$.

We have both $E \sqle{} D$ and $F \sqle{} F$.

There is no $x$ such that $x \sqlt{} B$ or $B \sqlt{} x$, only that $B \sqle{} B$.

\emph{Question:} Can you find all $x,y$ such that $x \sqle y$?

% Sometimes we also draw Hasse diagrams for infinite posets, hiding parts of the diagram with ellipsis. For example, the following is a Hasse diagram for the natural numbers.
% \begin{center}
%   \begin{tikzcd}[tips=false]
%     \vdots \ar{d} \\
%     2 \ar{d} \\
%     1 \ar{d} \\
%     0
%   \end{tikzcd}
% \end{center}

%%% Local Variables:
%%% mode: latex
%%% TeX-master: "../paper"
%%% End:
