\section{Chains and Heights}
\begin{definition}[Chain]
  If $(U, \sqle{})$ is a lattice, then $A \subseteq U$ is a \emph{chain} if:
  \begin{itemize}
  \item for all $x,y \in A$, either $x \sqle y$ or $y \sqle x$.
  \end{itemize}
\end{definition}

\begin{definition}[Height]
  If $(U, \sqle{})$ is a lattice, then the \emph{lattice height} is the carnality of the \textbf{longest} chain in the lattice.
\end{definition}

\begin{definition}[Ascending Chain Condition]
  A lattice (or more generally, a poset) satisfies the ascending chain condition
  if for any sequence $d_0 \sqle d_1 \sqle d_2 \sqle \cdots$, there exists an
  $m$ such that for all $n \geq m$ $d_n = d_m$.
\end{definition}

\emph{Question:} Can you have non-ascending chains in a finite lattice?

%%% Local Variables:
%%% mode: latex
%%% TeX-master: "../paper"
%%% End:
