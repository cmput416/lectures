\section{Partial Orders}
\begin{definition}[Partially Ordered Set, poset]
  If $U$ is a set and $\sqle$ is a binary relation on $U$, then the system
  $(U, \sqle)$ is a \emph{poset} if
  \begin{itemize}
  \item for every $x \in U$, $x \sqle x$ ($\sqle$ is reflexive),
  \item for every $x,y,z \in U$, if $x \sqle y$ and $y \sqle z$, then $x \sqle z$ ($\sqle$ is transitive), and
  \item for every $x,y \in U$, if $x \sqle y$ and $y \sqle x$, then $x == y$ ($\sqle$ is anti-symmetric)
  \end{itemize}
\end{definition}

\begin{example}[Posets]
  Some examples of posets include the following.
  \begin{itemize}
  \item $\leq$ on natural numbers.
  \item $\subseteq$ on sets.
  \end{itemize}
\end{example}

\begin{example}[Not posets]
  The following are not posets.
  \begin{itemize}
  \item $<$ on natural numbers - because $<$ is not reflexive. E.g. $1\nless{}1$.
  \item $\subset$ (proper subset) on sets - again not reflexive.
  \end{itemize}
\end{example}

%%% Local Variables:
%%% mode: latex
%%% TeX-master: "../paper"
%%% End:
